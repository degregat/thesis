\chapter{Background and Related Work}
\label{cha:background}

%\epigraph{Desvar\'io laborioso y empobrecedor el de componer vastos libros; el de explayar en quinientas p\'aginas una idea cuya perfecta exposici\'on oral cabe en pocos minutos.{Jorge Luis Borges,Pr\'ologo de Ficciones.}

%Los deterministas niegan que haya en el mundo un solo hecho posible, id est un hecho que pudo acontecer; una moneda simboliza nuestro libre albedrío. (€œEl Zahir€)

Learning what action to take to maximize a reward is a fundamental problem in decision theory. In it's modern form, using game theory to analize and guide the development of good strategies, is initiated with the same


We first provide a overview of the game theoretical background that underpins all aspects of this work.
% The literatures on bandits and decision markets have remained largely separate, 
We then present the building blocks from the two branches of it, bandit algorithms and mechanism design, that are used in our contributions.
We finalize the chapter by giving an overview of related work that we do not build upon but none the less informs or motivates our analysis.



\section{Notation and Conventions}


The notation used for is a compromise to accomodate as much as possible the conventions of both the bandit algorithms and mechanism design literatures. While adding the distinction between the action the algorithm or mechanism chooses and the one that the subject actually carries out in the world. We refer to throughout as the \emph{chosen} action and notate as $c$, and the action that the subject takes on the world, which we refer to this as the \emph{actual} action, and notate as $a$).

We follow the bandit literature and refer to rewards (notated as $r$ throughout) as directly observable after the actual action is taken. This is skipping the the maping of actions to outcomes, and the utility functions which map those outcomes to rewards which is standard in the mechanism design literature. The reader wishing to move the analysis more explciitly in the mechansim design tradition can replace the observed rewards with the von Neumann Morgensten utility function of the agent over the realized outcome.
%TODO: spell it out in greek
We treat experts signals, in the manner of features in a contextual bandit problem. To translate this to the formalism of the mechanism design literature, we can consider the cross product of the agents signals as defining a set of partitions, with one partition for each value. 
When we speak of two agents signals being identical, 

%The prior
%When taking expectations when it is not relevant we omit noting the prior $P$ explcitly. 

All models in the thesis use finite action spaces. In the strategic setting this gurantees the existance of Nash Eqilibria. 
ince actions involve reports of signals ($s$), this contraints all signals to also be discrete; note this does not constraint the underlying latent variables ($u$) to be discrete.



\section{Game Theory}

Statistical learning algorithms can be understood game theoretically as a game between a forecaster and nature.This is particularly natural in the sequential (online) setting, and a framework termed Learning with Expert Advice and \cite{cesa2006prediction} provide a unified treatment from a worst case game theoretic perspective of many such learning settings and algorithms.
This literature largely consider the underlying structures to be zero-sum and thus use a adversarial model of nature to construct strategies that have good worst case properties. In other words this is game theory in the style of Von Neumann and Morgensten 1948. The framework was applied to the setting of sequential experiments initially by Wald, and the bandit formalism introduced by \cite{robbins1952some}.

When there are multiple agents beyond nature interacting, as in elicitation from multiple experts for decision making, there are severe limits to what worst case analysis alone can yield. 
In particular, the notions of nash equilibrium \cite{nash1950equilibrium} and common knowledge \cite{aumann1976agreeing} provide a useful starting point to thinking about such settings, through they leave us with an embarrassment of riches in terms of the potetial equilibrium set.



\subsection{Online Learning}


The central object of analysis in the online learning (also known as learning with expert advice) framework is the \emph{regret} of an algorithm. This is defined as the diference in the cumulative reward between the algorithm and the reward that would have been obtained by some benchmark. 
The most common benchmark is the best fixed action in hindsight, and this is termed \emph{static regret}. When in this thesis we use the term regret without further modfiers, we are refering to this notion.

Two main settings appear in the literature for the play of the adversary is carried out: in the stochastic setting the adversary picks a distribution over actions at the start of the game form which IID draws are later made, in the oblivious adversary setting it selects a specific sequence of play before the game begins. Both choices are made with knowledge of the strategy of the learner, which is thus necesarily randomized in the oblivious adversarial case.

A crucial aspect of the expert advice setting is the feedback model; two fundamental extremes are full feedback, where after realizing a action the reward that would have been obtained for any other choice of action is also revealed, and the bandit setting where only the reward of the chosen action is revealed. 
A Partial observability graph Allows interpolation between full supervision and bandit feedback, shows sharp separations into three classes. 


\subsection{Mechanism Design}

How to structure a game so as to incentivize self interested agents to achieve some objective, is the central question of mechanism design.
Each agents information is characterized by their \emph{signal} (in the literature often also called their \emph{type} , particularly when it describes a private valuation of that agent of a good) which allows the agent to narrow down which realization of possible states of the world they are in. 

We say a mechanism is \emph{incentive compatible} when agents report their true signals to the mechanism. 
Without any further assumptions (i.e. without a probability distribution over said states of the world) signals are of limited use beyond situations with a dominant strategy equilibrium.
% TODO this needs a lot of refinement.
A very large literature exist on efficient mechanisms, particularly auctions, in the bayesian setting

Ideally one would like to search for mechanisms that are  \emph{dominant strategy} incentive compatible.
For many objectives of interest such mechanisms do not exist (as we will see in Chapter )


% % THERE EXISTS AN EXTENSIVE LITERATURE on efficient auctions and mechanism
% % design. A lot of attention has been devoted to the case where each agent i has a
% % quasi-linear utility function that depends on the chosen social alternative, on
% % information or signal privately known to  . i, and on a monetary transfer, but
% % does not depend on information available to other agents. 

% In this framework, a
% % prominent role is played by the Clarke-Groves-Vickrey CGV mechanisms see 
% % Clark 1971 , Groves 1973 , Vickrey 1961 . These are mechanisms that ensure 
% % both that an efficient decision is taken and that truthful revelation of privately
% % held information is a dominant strategy for each agent.
% %  The result holds for
% % %arbitrary dimensions of signal spaces and for arbitrary signals€™ distributions.
% % In this paper we study the case where each agent has a quasi-linear utility
% % function having as arguments signals received by all agents and the chosen
% % social alternative. Hence, besides allocative externalities, we allow for informational
% % %externalities, and we speak of €˜interdependent valuations. Signals may
% % be multi-dimensional, but we assume that they are independently drawn across
% % agents. Signal independence is the most seriously restrictive assumption;}



%http://users.eecs.northwestern.edu/~hartline/papers/bayesian-mechanism-design-2013.pdf
 bayesian mechanism design
% In the classic microeconomic treatment of mechanism design, the
% non-pointwise-optimality of mechanisms is resolved by formulating the
% problem as one of Bayesian design. A distribution over the preferences
% of the agents is given, and the designer seeks to optimize her objective in
% expectation over this distribution. For any specific distribution, there
% is such an optimal mechanism. The Bayesian assumption also allows
% the strategic incentives of the agents to be relaxed. Given that the
% agents’ preferences are drawn from a distribution, instead of requiring a
% mechanism to have dominant strategies, mechanisms can be considered
% where agents’ actions are best responses to the distribution of actions
% of other agents.
% Both the Bayesian and worse-case (henceforth, prior-free) frameworks
% for mechanism design have merits and, recognizing this, a
% Bayesian branch of algorithmic mechanism design has emerged.
% Paramount of study in Bayesian algorithmic mechanism design are
% algorithmic techniques, approximation, and computational issues.


% In Bayesian games it is useful to distinguish between stages of the
% game in terms of the knowledge sets of the agents. The three stages of
% a Bayesian game are ex ante, interim, and ex post. The ex ante stage
% is before values are drawn from the distribution. Ex ante, the agents
% know this distribution but not their own types. The interim stage is
% immediately after the agents learn their types, but before playing in
% the game. In the interim, an agent knows her own type and assumes
% the other agent types are drawn from the prior distribution conditioned
% on her own type, i.e., via Bayesian updating. In the ex post stage, the
% game is played and the actions of all agents are known.

% Proposition 2.1 (Revelation Principle, Myerson [74]). For any
% mechanism and Bayes-Nash (resp. dominant strategy) equilibrium of
% the mechanism, there exists a Bayesian (resp. dominant strategy) incentive
% compatible mechanism with the same equilibrium.
% The revelation principle fails to hold in some environments of
% interest. Two such environments, for instance, are where agents only
% learn their values over time, or where the mechanism designer does not know the prior and cant simulate.



\section{Bandits}

Decisions over mutually exclusive actions that affect reward naturally lead to partial supervision.
The 

Bandit problems are concerned with optimal repeated decision-making in the presence of uncertainty and partial supervision, while being simpler than full reinforcement learning in that they limit the underlying state of the system to evolve in a manner indepedent of the set of actions carried out. 

% make a note of the bandit process this being not strictly true as when you play an arm you allow the state in that arm to transition...

The main challenge is to trade-off exploration and exploitation, so as to collect enough samples to estimate the rewards from different potentially optimal actions whilst also strongly biasing samples towards those actions most likely to yield high rewards.  

 A prediction game that is central to the later part of this thesis is the contextual bandit setting: In each round, nature reveals some context, the adversary privately assigns a loss value to each action in a fixed set.
 Then the player chooses an action (possibly using randomization), incurs the corresponding loss, the player only observes the loss of the chosen action, but not that of other actions.
 The goal of the player is to control static regret, which is defined as the excess loss incurred by the player as compared to the best fixed action given the context over a sequence of rounds.

%Formally this can be defined as 
%TODO copy pasta terms from chapter

\subsection{Contextual Bandits}


% \subsection{Notation}


% For the purpose of this paper, we consider the multi-armed bandit problem with contextual information.  Following previous work~\cite{Langford08Epoch}, we call it a \emph{contextual bandit problem}.\footnote{In the
% literature, contextual bandits are sometimes called bandits with covariate~\cite{Woodroofe79One}, associative reinforcement learning~\cite{Kaelbling94AssociativeFunctions}, bandits with expert advice~\cite{Auer02Nonstochastic}, bandits with side information~\cite{Wang05Bandit}, and associative bandits~\cite{Strehl06Experience}.}
% %To make the framework more general (which our empirical study will adopt in
% %\secref{sec:application}), we also allow the set of arms to change over time.
% Formally, we define by $\Aset=\{1,2,\ldots,K\}$ the set of arms, and
% a contextual-bandit algorithm $\Aalg$ interacts with the
% \emph{world} in discrete trials $t=1,2,3,\ldots$.  In trial $t$:
% %
% In this process, the \emph{total $T$-trial payoff} of $\Aalg$ is defined as
% \[
% G_\Aalg(T) \defeq \E_D\left[\sum_{t=1}^T r_{t,a_t}\right],
% \]




% where the expectation $\E_D[\cdot]$ is defined w.r.t. the i.i.d. generation
% process of $(\vecb{x}_t,r_{t,1},\ldots,r_{t,K})$ according to distribution $D$
% (and the algorithm $\Aalg$ as well if it is not deterministic).
% Similarly, given a policy $\pi$ that maps contexts to actions, $\pi:\Cset\mapsto\Aset$, we define its total $T$-trial payoff by
% \[
% G_\pi(T) \defeq \E_D\left[\sum_{t=1}^T r_{t,\pi(\vecb{x}_t)}\right]=T\cdot\E_D\left[r_{1,\pi(\vecb{x}_1)}\right],
% \]
% where the second equality is due to our i.i.d. assumption.
% Given a reference set $\Pi$ of policies, we define the \emph{optimal expected $T$-trial payoff with respect to $\Pi$} as
% \[
% G^*(T) \defeq \max_{\pi\in\Pi} G_\pi(T).
% \]
% For convenience, we also define the per-trial payoff of an algorithm or policy,
% which is defined, respectively, by
% \begin{eqnarray*}
% g_\Aalg &\defeq& \frac{G_\Aalg(T)}{T} \\
% g_\pi &\defeq& \frac{G_\pi(T)}{T} = \E_D\left[r_{1,\pi(\vecb{x}_1)}\right].
% \end{eqnarray*}
% %
% Much research in multi-armed bandit problems is devoted to developing algorithms
% with large total payoff.  Formally, we may search for an algorithm
% minimizing \emph{regret} with respect to the optimal arm-selection
% strategy in $\Pi$.  Here, the $T$-trial regret $R_\Aalg(T)$
% of algorithm $\Aalg$ with respect to $\Pi$ is defined by
% %(\lihong{We may not need to define regret for this paper.})
% \begin{eqnarray}
% R_\Aalg(T) \defeq G^*(T) - G_\Aalg(T). \label{eqn:regret-def}
% \end{eqnarray}
% %
% An important special case of the general contextual bandit problem is the well-known \emph{$K$-armed bandit} in which the context $\vecb{x}_t$ remains constant for all $t$.  Since both the arm set and contexts are constant at every trial, they have no effect on a bandit algorithm, and so we will also refer to this type of bandit as a \emph{context-free} bandit.


\subsection{Algorithms}

We now provide a brief overview of two algorithms that are important buidling blocks in our later analysis. 

\subsubsection{Thompson Sampling}

 A very natural algorithm with good practial performance and is first proposed in the literature by \cite{thompson:33}, play each action with probability equal to its posterior probability (given the rewards observed up to that point) of being the best action.



% %  The
% % general structure of TS involves the following elements
% % (this description of TS follows closely that of
% % Chapelle and Li [2011]):
% % 1. a set ψ of parameters ˜µ;
% % 2. an assumed prior distribution P(˜µ) on these parameters;
% % 3. past observations D consisting of (reward r) for
% % the arms played in the past time steps;
% % 4. an assumed likelihood function P(r|µ˜), which
% % gives the probability of reward given a context
% % b and a parameter ˜µ;
% % 5. a posterior distribution P(˜µ|D) ∝ P(D|µ˜)P(˜µ),
% % where P(D|µ˜) is the likelihood function.

% % The notation P(·) in above denotes probability density.
% % TS maintains a posterior distribution for the underlying
% % parameter µi
% % , i.e. the expected reward, of every
% % arm i. In each round, TS plays an arm according to
% % its posterior probability of being the best arm, that is,
% % the posterior probability of having the highest value of
% % µi
% % . A simple way to achieve that is to produce a sample
% % from the posterior distribution of every arm, and
% % play the arm that produces the largest sample. Below
% % we describe two versions of TS, using Beta priors
% % and Bernoulli likelihood function,

% and using Gaussian
% % priors and Gaussian likelihood respectively.
% % We emphasize that the Beta priors and Bernoulli likelihood
% % model, or Gaussian priors and the Gaussian likelihood
% % model for rewards are only used below to design
% % the Thompson Sampling algorithm. Our analysis of
% % these algorithms allows these models to be completely
% % unrelated to the actual reward distribution. The assumptions
% % on the actual reward distribution are only
% % those mentioned in Section 1.1, i.e., the rewards are in
% % the range [0, 1]. I



\subsubsection{Exp3}

 Maximally robust guarantees, in particular wrt non-stochastic nature of underlying sequences. Useful when we wish to create hierarchical bandits when the original sequence may not be IID, the sequence that results from a bandits algorithms choices will not be by construnction.

%Beyond IID and Static Regret: E

\section{Information Aggregation and Incentives}

A literature in economics and particularly mechanism design, is centered on when and how information can be aggregated from multiple agents that receive signals about the state of the world, and have various degrees of strategic sophistication in their actions.

explain myopic incentive compatible


% \subsection{Information Revelation and Aggregation in Markets}
% TODO: rewrite 
% %\quote{The question of information revelation and aggregation in markets has attracted the attention of many economists, beginning with Hayek (1945). Grossman (1976) formally %showed that in a market equilibrium, the result- ing price aggregates information dispersed among n-types of informed traders, each of whom gets a €œpiece of information.€ In his model, individual traders are small relative to the market, strategic foundations for players€™ behavior are lacking, and the results rely on particular functional forms (e.g., i.i.d. normal errors in signals received by the players; normal prior; etc.). Radner (1979) introduced the concept of Rational Expectations Equilibrium (REE) and showed that generically, a fully revealing REE exists, with prices aggre- gating all information dispersed among traders. Radner€™s paper, however, also lacks strategic foundations. A series of papers explored the question of con- vergence to REE in various dynamic processes (see, e.g., Hellwig (1982), and Dubey, Geanakoplos, and Shubik (1987), for models of centralized trading; Wolinsky (1990), and Golosov, Lorenzoni, and Tsyvinski (2011), for models of decentralized trading; and McKelvey and Page (1986), and Nielsen, Bran- denburger, Geanakoplos, McKelvey, and Page (1990), for models that extend the basic communication process of Geanakoplos and Polemarchakis (1982) to more complex settings in which agents€™ beliefs are iteratively updated in re- sponse to repeated public observations of summary statistics of their actions). In all of these papers, however, it is assumed that each trader ignores the ef- fect of his behavior on the evolution of the trading process, as a result behaving non-strategically along at least one dimension. P

% % roper strategic foundations for the concept of perfect competition with differentially informed agents are offered by the stream of literature studying bidding behavior in single and double auctions (Wilson (1977), Milgrom (1981), Pesendorfer and Swinkels (1997), Kremer (2002), Reny and Perry (2006)). Information aggregation results in these papers, however, rely on the assumption that the market is large, that is, the number of bidders goes to infinity and individual traders become small relative to the market. They also rely on various symmetry assumptions. No such assumptions are made in the current paper, and the number of traders is finite and fixed.

% % Kyle (1985) offered a model of dynamic insider trading, in which the single informed trader takes into account the nonnegligible impact of his actions on market prices. In the continuous version of the model, as time approaches the end of the trading interval, the price of the traded security converges to its true value known by the insider. Foster and Viswanathan (1996) and Back, Cao, and Willard (2000) extended the model to the case of multiple, differentially informed strategic traders. In the continuous case, the price of the traded secu- rity converges to its 
% %expected value conditional on the traders€™ pooled informa- tion. In the discrete case with a finite number of trading periods, convergence is approximate. These models rely on very special functional form assumptions (symmetry, normality, etc.), which allow the authors to construct explicit for- mulas for particular (€œlinear€) equilibria. Laffont and Maskin (1990) criticized this reliance of the results of Kyle (1985) on linear trading strategies; argued that such models inherently have multiple equilibria; presented a model of a trading game with a single informed trader and multiple equilibria, in some of which the informed trader€™s information is not revealed; and concluded that €œin a model in which private information is possessed by a trader who is big enough to affect prices, the information efficiency of prices breaks down€ and €œthe efficient market hypothesis may well fail if there is imperfect competi- tion.€ The results of the current paper show that the conclusions of Kyle (1985), Foster and Viswanathan (1996), and Back, Cao, and Willard (2000) regarding the convergence of the price of a security to its expected value conditional on the traders€™ pooled information do not, in fact, depend on the specific func- tional form assumptions or on the choice of equilibrium: if the traded security is separable, its price converges to its expected value conditional on the pooled information in every equilibrium. In the case of a single informed trader, as in Laffont and Maskin (1990), every security is separable, and so information always gets aggregated. The contrasting conclusions of Laffont and Maskin are thus driven by the specifics of their model, not by the fact that they explicitly consider multiple equilibria. In the case of multiple partially informed traders, the securities considered in Foster and Viswanathan (1996) and Back, Cao, and Willard (2000) have payoffs that are linear in traders€™ signals, and so, as the results of this paper show, information about such securities always gets aggregated as well.

% % This model is based on the market scoring rule (MSR) of Hanson (2003, 2007). In MSR games, there are no noise or liquidity traders and no strategic market makers; the only players are the strategic par- tially informed traders. There is also an automated market maker. This mar- ket maker, in expectation, loses money (though at most a finite, ex ante known amount), %facilitating trade and price discovery. (Without a €œsource€ of prof- its, there would be no trading; see Milgrom and Stokey (1982) and Sebenius and Geanakoplos (1983).) Trading 
% % proceeds as follows. The uninformed mar- ket maker makes an initial, publicly observed prediction about the value of a security. The first informed strategic player can revise that number and make his own prediction, which is also observed by everyone. Then the second player can further modify the prediction, and so on until the last player, after which the first player can again modify the prediction, and the cycle repeats an infinite number of times. The fact that there is an infinite number of trading periods does not mean that the game never ends. %Rather, it is a convenient discrete analogue of continuous trading, with trades taking place at times t0 < t1 < · · · in a bounded time interval. Sometime after the trading is over, the true value of the security is revealed, and each prediction is evaluated according to a strictly proper scoring rule s (e.g., under the quadratic scoring rule, each prediction is penalized by the square of its error; see Section 2.2 for further details). The payoff of a player from each revision is the difference between the score of his prediction and the score of the previous prediction€”in essence, the player €œbuys out€ the previous prediction and replaces it with his own. The total pay- off of a player in the game is the sum of payoffs from all his revisions. Players are risk-neutral. The discounted MSR (Dimitrov and Sami (2008)) is similar, except that the total payoff of a player is equal to the discounted sum of pay- offs from all his revisions, where the payoff from a revision made at time $t_k$ is multiplied by $I^2k$ for some $ i< 1$.
% % While my primary reason for studying this model is to illustrate the intu- ition behind information aggregation in the main model and thus make that result more transparent, information aggregation in MSR-based models is also of independent interest, for several reasons. First, such models can be viewed as generalizations of the communication processes of Geanakoplos and Pole- marchakis (1982) and other papers in this tradition, in which several differen- tially informed agents sequentially announce their beliefs about the value of a random variable (or the probability of an event), and those beliefs eventually
% % converge to a common posterior. In those papers, it is assumed that the agents make truthful announcements, and strategic issues are ignored. Discounted MSR includes this truthful process as a special case, I= 0 (strictly speaking, the case I = 0 is ruled out in this paper, but it is easy to show that as I becomes small, in any equilibrium, players will behave almost %myopically, i.e., will reveal their expectations almost truthfully), and at the same time makes it possible to examine the role of strategic behavior (for β > 0). The results of this paper show that, for separable securities, information aggregation does not depend on whether agents behave strategically or myopically.
% % Second, the MSR model includes as a special case a basic model of trading
% % with an automated inventory-based market maker who offers to buy or sell
% % shares in the security at price p that is a function of the (possibly negative) net
% % inventory the market maker holds at that moment. Specifically, suppose the
% % market maker starts with zero net inventory, sets the price for the security as
% % a continuous decreasing function p(z), where z is the total amount of shares
% % he holds in his inventory (i.e., the more he holds, the less he is willing to pay
% % for additional shares), and commits to buying or selling shares according to
% % that price schedule. Thus, if his current inventory is z0, and a trader decides
% % %to sell (z ˆ’ z ) units of the security to the market maker, the market maker 1 0 ô°‹z1
% % %will pay that trader z0 p(z ̃)dz ̃ for the (z1 ˆ’z0) units. The current price of the security will move from p(z0) to p(z1). If the true value of the security then
% %turns out to be equal to x, then the trader€™s payoff from this transaction will be equal to ô°‹ z1 p(z ̃) dz ̃ ˆ’ x(z1 ˆ’ z0) = ô°‹ z1 (p(z ̃) ˆ’ x) dz ̃ô°· Thus, it is strictly optimal
% %z0 z0
% %(myopically) for the trader to pick z1 in such a way that p(z1) is equal to his
% %belief about the value of the security (assuming the image of function p(·)
% %includes that value). His payoff from this transaction is equal to his payoff from
% s%coring rule s(yô°¸x)= 0 (p(z ̃)ˆ’x)dz ̃, where zy =pˆ’1(y), that is, p(zy)=y.2 Finally, note that while the market maker in this setting expects to lose money, the worst possible loss is bounded and can be controlled by adjusting the parameters of the rule. Another attractive feature of MSR in practice (rel- ative to, say, continuous double auctions) is that a player can instantaneously make his prediction/trade at any time, without having to wait for another player who is willing to take the other side of the trade or to submit a limit order and wait for it to be filled. These features make MSR attractive for use in internal corporate prediction markets, and it is in fact used for that purpose: compa- nies like Consensus Point and Inkling Markets operate MSR-based prediction
% % markets for Ford, Chevron, Best Buy, General Electric, and many other large corporations.3 Thus, the question of whether information in MSR-based pre- diction markets gets aggregated has direct practical implications.
% % Two recent papers have studied the equilibrium behavior of traders in MSR games.4 Chen, Reeves, Pennock, Hanson, Fortnow, and Gonen (2007) consid- ered undiscounted %games based on a particular scoring rule€”logarithmic (see Section 2.2). In their model, the security can take one of two different values, and the number of revisions is finite. They found that if traders€™ signals are in- dependent conditional on the value of the security, then it is an equilibrium for each trader in each period to behave myopically, that is, to make the predic- tion equal to his posterior belief. They also provided an example of a market in which signals are not conditionally independent and one of the traders has an incentive to behave non-myopically. Dimitrov and Sami (2008) also considered games based on the logarithmic scoring rule. In their models, in contrast to Chen et al., traders observe independent signals. Each realization of the vector of signals corresponds to a particular value of the security. The number of trad- ing periods is infinite. Dimitrov and Sami found that, in that case, in the MSR game with no discounting, myopic behavior is generically not an equilibrium and, moreover, there is no equilibrium in which all uncertainty is guaranteed to get resolved after a finite number of periods. They then introduced a two- player, two-signal MSR game with discounting, and proved that in that game, information gets aggregated in the limit, under the additional assumption that the €œcomplementarity bound€ of the security is positive. They reported that, based on their sample configurations, the bound is not always zero, but did not provide any sufficient conditions for it to be positive. In contrast to Chen et al. (2007) and Dimitrov and Sami (2008), the current paper€™s information aggregation results (1) do not rely on the independence or conditional inde- pendence of signals, allowing instead for general information structures with any number of players; (2) do not depend on discounting; and (3) provide a sharp characterization of securities for which information always gets aggre- gated and those for which, under some priors, price may not converge to the expected value conditional on the traders€™ pooled information.}

% %Ostrovsky studies information aggregation in dynamic markets with a finite number of partially informed strategic traders. It shows that, for a broad class of securities, information in such markets always gets aggregated. Trading takes place in a bounded time interval, and in every equilibrium, as time approaches the end of the interval, the market price of a €œseparable€ security converges in probability to its expected value conditional on the traders€™ pooled information. If the security is €œnon-separable,€ then there exists a common prior over the states of the world and an equilibrium such that information does not get aggregated. The class of separable securities includes, among others, Arrow€“Debreu securities, whose value is 1 in one state of the world and 0 in all others, and €œadditive€ securities, whose value can be interpreted as the sum of traders€™ signals.

% %Consider the following example from Geanakoplos and Polemarchakis (1982).
% %EXAMPLE1: Therearetwoagents,1and2,andfourstatesoftheworld,Ω=
% %{Aô°¸Bô°¸Cô°¸D}. The prior is P(ω) = 1 for every ω ˆˆ Ω. The security is X(A) = 4
% %X(D) = 1 and X(B) = X(C) = ˆ’1. Partitions are Π1 = {{Aô°¸B}ô°¸{Cô°¸D}} and Π2 ={{Aô°¸C}ô°¸{Bô°¸D}}.

% %In the example, by construction, it is common knowledge that each player€™s expectation of the value of the security is zero, even though it is also common knowledge that the actual value of the security is not zero, and that the traders€™ pooled information would be sufficient to determine the security€™s value. Thus, even if the traders repeatedly and truthfully announce their posteriors, as in Geanakoplos and Polemarchakis (1982), they will never learn the true value of the security. 





\subsection{Prediction Markets} 

The closest contact point between the online learning and information elicitation literatures is in the fully supervised case.
That is the information the market is attempting to aggregate is a forecast of the future state of the world that is not contigent on the actions that can be influencd by it. Thus, at the time of realization of the event, we can judge not only the forecast obtained but also any other potential forecast that ould have been received. 
This contrasts this with the bandit setting, where instead of a forecast of the state of the world we seek an action that results in a state of the world that is maximally beneficial.  


The equivalence between trading shares and eliciting beliefs from a single agent by the means of scoring rules goes back at least to \cite{savage1971elicitation}. 
%Hanson (2003), Pennock (2006), and Chen and Pennock (2007) discussed this cor- respondence for the case of MSR. The study of automated market makers goes back to Black (1971a, 1971b), while formal analysis of inventory-based market makers goes back to Amihud and Mendelson (1980).



% %The correspondence between trading shares and eliciting beliefs from a single agent by the means of scoring rules was first noted by Savage (1971), who also provided additional techni- cal details. 
% %

% %Under the basic MSR (introduced by Hanson (2003, 2007), though the idea of repeatedly using a proper scoring rule to help forecasters aggregate infor- mation goes back to McKelvey and Page (1990))
% \cite{mckelvey1990public}
% %Note also that if each player behaves myopically in each period, the prediction that he will make is his posterior belief about the expected value of the security, given his initial information and the history of revisions up to that point, and thus the €œgame€ turns into the communication process of Geanakoplos and Polemarchakis (1982).

Initiating with the equivalence between market scoring rules and regularized follow the leader algorithms in \cite{chen2010new} and series of follow up works \cite{abernethy2013efficient, frongillo2012interpreting, hu2014multi, frongillo2015convergence,} that map prediction markets to learning algorithms. 

Subjects freedom makes no difference in the analysis of these fully supervised setting; since there is no action to take, there is no sense in which a subject may not follow along. To the degree the information they surface is not being used strategically by the agents, we can perfectly evaluate how accurate they are regardless of the other agents reports.
% note this is not the case in our ex-comp one shot example, where to evaluate one experts acuracy we need the truthful signal from the other 




%This assumption seems to largely have held for sports and politics, and betting markets over both 

Another notable difference between the decision and 

Learning performance of prediction markets with Kelly bettors\cite{beygelzimer2012learning}

\subsection{Information Design and Incentive Compatible Bandits}

% Information Design, Bayesian Persuasion, and Bayes Correlated Equilibrium,  American Economic Review Papers and Proceedings, 2016, 106, 586-591, with Stephen Morris


 \cite{mansour2015bayesian}


% complexity theory of algorithmic persuasion
%http://arxiv.org/pdf/1503.05988.pdf



\subsection{Decision Markets}

Contracts that pay if a publicly verifable event occurs are generally regulated as effectively gambling,  severely restricts their use. 

A small  niche of providers of automated market making software for private (particularly corporate outcomes)



\cite{berg2003prediction,hanson2002decision,othman2010decision,boutilier2012eliciting,chen2014eliciting}

% \subsection{Decision Markets}

%Prediction markets have served as a reliable tool for estimating the winners of political elections and sports games [Berg et al., 2001]. However, due to legal restrictions severely limiting their use, the latest wave of prediction markets have focused not on the general, but on the specific. Instead of creating large-scale markets
% % on publicly verifiable events, these services bill themselves as accu- mulators of organizational knowledge on specific internal (e.g., cor- porate) events. For instance, the prediction 
% % %market startup Inkling Markets suggests that companies use their service €œto uncover and quantify risk in your organization". The recently launched Crowd- cast lists questions like €œWhen will your product really ship?" and €œHow much will it really cost?" on their homepage.

\cite{othman2010decision} argue that corporate prediction markets do not capture the right problem for their clients.
In particular, by focusing on eliciting probabilities after decisions has been made about what their effects will be, they cannot be used to inform those decisions. 

% % We contend that the current corporate prediction markets do not actually capture the problems most prominent for their clients. The chief challenge facing businesses is not whether the decisions they have made will prove successful, but rather what actions to take to assure the best chance of success.
% % %Our work approaches this issue directly. 
%In our model, a princi- pal has to choose an action from a set of possibilities (e.g., €œHire additional sales staff", €œDouble R&D funding") in order to maxi- mize the probability of achieving a desirable outcome (e.g., €œBe- come the top-grossing company in our market space", €œSell more than a million widgets", or €œAchieve profitability by the end of the year"). The principal elicits from an expert, for each action, the probability of achieving the desired outcome. Based on those prob- abilities, the principal then (deterministically) chooses an action ac- cording to a decision rule. Upon success or failure in achieving the desired outcome, the principal rewards the expert according to a pre-determined scoring rule. Three functions define the space: the decision rule, the payoff for success, and the payoff for failure.
% % We first prove general properties of the way these functions re- late. We then provide an in-depth study of the most natural de- terministic decision rule, the max decision rule, which selects the action that has the highest reported success probability. For the single-agent setting, we show that no symmetric scoring rule, nor the asymmetric ones from the literature, give the agent the right in- centives. We construct an asymmetric scoring rule that does, and provide a characterization of a space of such rules.
% % Along the same lines as the original construction of market scor- ing rules [Hanson, 2003, 2007, Pennock and Sami, 2007], we con- tinue by attempting to expand our scoring rule system into an auto- mated market maker for our multiagent setting, to create decision markets. Surprisingly, we show that every market of this kind suf- fers from a peculiar type of manipulability, where an agent benefits from exaggerating the success probability of a suboptimal action. We show that this kind of manipulability applies even under an in- finite stream 
% % %of self-interested agents visiting the market€”even if each agent€™s beliefs about the probabilities are exactly accurate. We design families of asymptotically-optimal pricing rules for decision markets that minimize manipulability.
% % Finally, we study two alternate decision market designs. We show the first one suffers from significant manipulability of a dif- ferent kind, and for the second we show a new kind of no-trade result. 	
% % We initiated the study of decision rules and decision markets in settings where a principal needs to select an action and is advised by a self-interested but decision-agnostic expert about the success probabilities of alternative actions.
% % %We began by investigating the properties of general deterministic decision rules in the context of eliciting from a single expert. We proved results about the relations between the principal€™s decision rule and the rules that specify the expert€™s payoff if the desired out- come is, and is not, achieved. For the most natural decision rule (where the principal takes the action with highest success probabil- ity), we showed that no symmetric scoring rule, nor any of Win- kler€™s asymmetric scoring rules, are quasi-strictly proper. We char- acterized the set of differentiable quasi-strictly proper scoring rules and constructed an asymmetric scoring rule that is quasi-strictly proper.
% % Moving to decision markets where multiple experts interact by trading, we showed a surprising impossibility for every automated market maker, where an agent is incentivized to artificially raise the price of a non-optimal action (again under the decision rule where the principal takes the action with highest success probability). To counter this impossibility, we constructed two families of asymp- totically optimal pricing rules against this form of manipulation, one additively optimal and the other multiplicatively optimal. Fi- nally, we considered two alternative market designs for decision markets. The first, in which all outcomes live in the same proba- bility universe, has even worse incentives (for the final participant). The second, in which the experts trade on the probability of the outcome occurring unconditionally, exhibits a new kind of no-trade result.
% % There are many interesting research directions in this new area.
% % The first would be examining the space of discontinuous scoring rules, being aware that such rules would nevertheless have to obey Theorem 2. Another area of exploration would be combinatorial decision rules, in which the principal could take more than one of the available actions.
% % We focused on deterministic decision rules because they are nat-
% % ural. It is known that adding randomness can significantly increase
% % %the incentive-compatible feasible space of mechanisms in other settings€” prominent examples include voting [Gibbard, 1977] and sybill-proof (false name resistant) voting [Wagman and Conitzer, 2008]. Future research should study randomized decision rules in our setting as
% % %well. Trivial randomized solutions €”like shifting probability to
% % %each action do not get around our impossibilities, like the nonex- istence of strictly proper scoring rules (because the manipulations
% % in our examples and proofs yield the manipulator benefits that are
% % not infinitesimally small).
% % It would also be interesting to explore a larger characterization of the peculiar no-trade impossibility of Section 3.4.2. Does it apply in other economic interactions besides markets of this sort? The key seems to be that any action an agent takes has negative utility because of the effects of taking that action.
% % %Another important direction concerns what happens when agents are interested in the action taken by the principal. For instance, an expert could advocate doubling a company€™s advertising bud- get because she works in the marketing department. This type of setting is related to recent work by Shi et al. [2009], who study a setting where an agent can perform an action after a market runs such that the market can incentivize the agent to act counter to the principal€™s goal. Whether or not the agents can take actions after the decision market, it would desirable to use decision markets to balance the utilities of agents impacted by the principal€™s decision with the principal€™s desire to achieve his goal.}




% We address this manipulation and construct and characterize decision markets
% that, in expectation, always score predictions for their accuracy. Markets
% with this property are myopically incentive compatible, since participants always
% maximize their score for a prediction by honestly revealing their private beliefs.
% Instead of a scoring rule, these markets use a decision scoring rule that accounts
% for the likelihood actions are taken when scoring predictions. When the decision
% maker risks taking an action at random, a decision scoring rule can amplify
% the scores of unlikely actions and reduce the scores of likely ones, making risk
% neutral participants indifferent to their affects on the decision. We show this
% willingness to take an action at random is a requirement for these markets, and
% reducing this risk increases the decision maker’s potential loss. While we do not
% directly discuss information aggregation in decision markets, we demonstrate a
% correspondence between their perfect Bayesian equilibria and those of a set of
% prediction markets.

\section{Related Work}


\subsection{Mechanisms and the Price of Anarchy}

The online learning and mechanism design branches of game theory are not entirely disjoint. 

Ideally one would like mechanisms that have good equilibria in the sense of efficiency and truthfulness, where learning dynamics can help arrive on those equilibria. 
In a sense, dominant strategy and BNE are the limits of purely deductive learning, since the optiomal action can be deduced purely from the structure of the game before it is played. 
In the case of the dominant strategies it is robust (in the minimax sense) to the motivation and information of the other agents, this also limits the degree of efficiency it can achieve. 
Bayes Nash Equilibria on the other hand, suffer the opposite problem, with priors are common knowledge then there are equilibria that are fully revealing and incentive compatible (by the revelation principle), but can be many other equilibria where nothing is learnt; being inherently nonconstructive BNE cant help us distinguish between these. 


Methodologies from the literature on  the price of anarchy can be used to bypass this in some situations.
This style of analysis does not explicitly sole for equilibrium, instead it tries to characterize the minimal properties of a equilibria to quantify the worst case loss of any such equilbria relative to the optimum.

% considers minimal necessary properties of equilibria to argue that any
% equilibrium must be pretty good. Of course, this would be a problematic
% exercise if we believed that, just as we cannot solve for equilibrium,
% neither can the agents. Fortunately, because the analysis employs only
% minimal assumptions, a frequent corollary of a price-of-anarchy analysis
% is that best-response dynamics and no-regret learning algorithms
% have good performance even if they never reach an equilibrium.
% To illustrate the power of the price-of-anarchy approach, we will
% describe a few recent results. Consider the welfare objective in multiagent
% multi-item environment (which we consider in Section 5 and
% Section 6 for the revenue objective). Suppose instead of running a
% simultaneous auction that coordinates the sale of the items, we run
% independent first- or second-price auctions either simultaneously or
% sequentially. These auctions are strategically complex, e.g., because
% there is an exposure problem where an agent may win multiple items
% even if she only wants one. A series of papers, Bikhchandani [13],
% Christodoulou et al. [34], Bhawalkar and Roughgarden [12], Hassidim
% et al. [63], Paes Leme and Tardos [78], Paes Leme et al. [79], Roughgarden
% [83], and Syrgkanis and Tardos [88, 89] showed that in various
% configurations the price of anarchy of independent auctions for multiple
% items is often a constant like two. Paes Leme et al. [77] give a short
% survey of these results.
% Price of anarchy approaches have also been applied to an auction
% known as the generalized second-price auction which is used by Internet
% search engines to sell advertisements that are displayed alongside search
% results (see, e.g., Fain and Pedersen [46]). In this auction, bidders are
% ranked by bid and each bidder is charged the bid of her successor. This
% auction does not have simple equilibria as does the second-price auction.
% Gomes and Sweeney [51] show that even in symmetric Bayesian
% settings the generalized-second-price auction may not have any effi-
% cient equilibria; subsequent work of Caragiannis et al. [25] bounded the
% potential inefficiency of its equilibria by a factor of slightly under three.
% \section{Composable Mechanisms and Learning Reductions}

% A final class of results looks at non-revelation mechanisms based
% on approximation algorithms (because, as mentioned above, there are
% no general approaches for converting approximation algorithms into
% mechanisms without Bayesian assumptions). Lucier and Borodin [71]
% consider multi-dimensional combinatorial auctions based on greedy
% approximation algorithms. They show that in equilibrium, a mechanism
% based on a greedy β approximation algorithm is at worst a (β + 1)
% approximation in equilibrium.
% The results described above are primarily for the welfare objective,
% and methodologies from the price of anarchy have seen relatively less
% success for the revenue objective. One exception is by Lucier et al. [72]
% who give revenue bounds for the previously discussed generalizedsecond-price
% auction.


% The conventional approach to the analysis of Bayes-Nash equilibrium,
% as a first step, explicitly solves for the Bayes-Nash equilibrium. For
% asymmetric environments such an analysis would require the solution to
% analytically intractable differential equations. The approximation-based
% approach presented herein circumvents solving for BNE by decomposing
% the analysis into the following two parts. The first part isolates the
% best-response property of Bayes-Nash equilibrium and formalizes the
% intuition that either an agent gets good utility or must be facing fierce
% competition. The second part identifies a revenue covering property, that
% revenue exceeds an aggregate measure of the competition faced by each
% agent, as a criteria to be approximated. With bounds on utility and revenue,
% we get approximation bounds on the social surplus (the sum of
% utility and revenue).
% The bounds we derive on the social surplus and revenue of auctions
% in Bayes-Nash equilibrium are parameterized by the extent to which
% revenue covering approximately holds. This observation then gives clear
% direction for optimization in mechanism design. A Bayes-Nash mechanisms’s
% performance is proportional to it approximation with respect to
% revenue covering. Bayes-Nash mechanisms should be designed to minimize
% this approximation



% Composable and Efficient Mechanisms, Vasilis Syrgkanis, Eva Tardos

% meets

% Learning Reductions that Really Work, Mineiro

% \section{Applications of Adaptive Designs and Predicton Markets}


% %A. Kolotilin. Experimental design to persuade. UNSW Australian School of Business Research Paper,


% There is to be best of our knowledge no work that previously considers incentives for experts who provide the contextual variables in a contextual bandit problem. Beyond the already mentioned examples of considering the incentives for subjects in bandits, there is little work that brings incentives into the learning with expert advice setting. 


% The finl chapters concern with a sequence of decisions, related to this are the properties of repeating one shot seccond price auctions, which has been previously analized in terms of the composability of auctions. 

% Smooth Games, Price of Anarchy and
% Composability of Auctions - a Quick Tutorial
% Abhishek Sinha
% Laboratory for Information and Decision Systems, Massachusetts Institute of Technology, Cambridge, MA 02139
% Email: sinhaa@mit.edu
% Abstract
% In this tutorial we review the concept of smooth games as introduced by Roughgarden in [1]. We show how to upper bound
% the Price of Anrchy of different games using this general tool. Then along the lines of [2], we extend the idea of smooth games
% to smooth mechanisms and weakly smooth mechanisms and prove that many of the common auctions fall under this category. We
% then show how efficiency results of these mechanisms in a Bayes-Nash equilibria may be derived from smoothness arguments.
% Then we prove that simultaneous composition of smooth mechanism preserves the smoothness property. Hence all the previous
% results about efficiency of smooth mechanisms carry over to simultaneous compositions.




% \subsubsection{Efficiency}
% Taken from 
% % \quote{
% % EFFICIENT DESIGN WITH INTERDEPENDENT VALUATIONS
% % BY PHILIPPE JEHIEL AND BENNY MOLDOVANU1
% % We study efficient, Bayes-Nash incentive compatible mechanisms in a social choice
% % setting that allows for informational and allocative externalities. We show that such
% % mechanisms exist only if a congruence condition relating private and social rates of
% % information substitution is satisfied. If signals are multi-dimensional, the congruence
% % condition is determined by an integrability constraint, and it can hold only in nongeneric
% % cases where values are private or a certain symmetry assumption holds. If signals are
% % one-dimensional, the congruence condition reduces to a monotonicity constraint and it
% % can be generically satisfied. We apply the results to the study of multi-object auctions, and
% % we discuss why such auctions cannot be reduced to one-dimensional models without loss
% % of generality.


% % THERE EXISTS AN EXTENSIVE LITERATURE on efficient auctions and mechanism
% % design. A lot of attention has been devoted to the case where each agent i has a
% % quasi-linear utility function that depends on the chosen social alternative, on
% % information or signal privately known to  . i, and on a monetary transfer, but
% % does not depend on information available to other agents. In this framework, a
% % prominent role is played by the Clarke-Groves-Vickrey CGV mechanisms see 
% % Clark 1971 , Groves 1973 , Vickrey 1961 . These are mechanisms that ensure 
% % both that an efficient decision is taken and that truthful revelation of privately
% % held information is a dominant strategy for each agent.
% %  The result holds for
% % %arbitrary dimensions of signal spaces and for arbitrary signals€™ distributions.
% % In this paper we study the case where each agent has a quasi-linear utility
% % function having as arguments signals received by all agents and the chosen
% % social alternative. Hence, besides allocative externalities, we allow for informational
% % %externalities, and we speak of €˜interdependent valuations. Signals may
% % be multi-dimensional, but we assume that they are independently drawn across
% % agents. Signal independence is the most seriously restrictive assumption;}


% The proof of Theorem 4.3 is based on the following technical observation: an
% incentive compatible mechanism generates for each agent a vector field. This
% field associates to each type a vector of expected probabilities with which the
% various alternatives are chosen. A generalization of the standard one-dimensional
% envelope argument shows that this vector field is the gradient of the
% equilibrium expected utility function. Since it is a gradient, the vector field must
% satisfy an integrability condition involving its cross-derivatives.11 The impossibility
% results follow by showing that the vector fields generated by efficient
% mechanisms satisfy the required conditions only under very restrictive conditions.
% Since the integrability constraint bites in any multi-dimensional model, results
% similar to Theorem 4.3 hold as soon as there is at least one agent whose signal is
% of dimension d2.
% In Section 5 we study the remaining case where signal spaces are one-dimensional.
% We construct a mechanism that is efficient and incentive compatible if
% several inequalities relating private and social marginal valuations are satisfied.
% The main idea of the construction is to make i’s transfer equal to the cumulative
% Ž . 12 effect of i’s action here a signal report on all other agents. Since i’s effect on
% others depends here on i’s signal, incentive compatible transfers must neutralize
% this influence.



% \subsection{Decisions}

% \quote{If we relax the need to elicit probabilities, the problem can be solved straightforwardly. For instance, it is trivial to incentivize an agent to truthfully report which action has the highest probability of achieving a desired outcome. This can be done by simply giving the agent a lump sum payment if the outcome is reached, which aligns the incentives of the agent and the principal.} \cite{othman2010decision}





% \cite{dreber2015using}
% use a prediction market to estimate the reproducibility of scientific research


% a incromprehensible paper, but it is a trivial corolary of the folk theorem (TODO: cites beyond Della Penna and Reid 2012, Chen and ) is that any mechanism that settles before the event occurs has the multiple equilibria. Further if one agent has larger cpaital than the others he can force the settlement to favor him (you might be right, but he is more liquid) %http://bjll.org/index.php/jpm/article/view/1148


% \section{}
% % %http://arxiv.org/pdf/1507.00407v5.pdf
% % It is natural in this setting for the players to each make use of a no-regret learning algorithm for making
% % their decisions, an approach known as decentralized no-regret dynamics. No-regret algorithms
% % are a strong match for playing games because their regret bounds hold even in adversarial environments.
% % As a benefit, these bounds ensure that each player’s utility approaches optimality. When
% % played against one another, it can also be shown that the sum of utilities approaches an approximate
% % optimum [2, 19], and the player strategies converge to an equilibrium under appropriate conditions
% % [7, 1, 9], at rates governed by the regret bounds. Well-known families of no-regret algorithms
% % include multiplicative-weights [14, 8], Mirror Descent [15], and Follow the Regularized/Perturbed
% % Leader [13]. (See [3, 21] for excellent overviews.) For all of these, the average regret vanishes at
% % the worst-case rate of O(1/
% % √
% % T), which is unimprovable in fully adversarial scenarios.
% % However, the players in our setting are facing other similar, predictable no-regret learning algorithms,
% % a chink that hints at the possibility of improved convergence rates for such dynamics. This
% % was first observed and exploited by Daskalakis et al. [4]. For two-player zero-sum games, they developed
% % a decentralized variant of Nesterov’s accelerated saddle point algorithm [16] and showed
% % that each player’s average regret converges at the remarkable rate of O(1/T). Although the resulting

% % We consider the setting where the game G is played repeatedly for T time steps. At each time
% % step t each player i picks a mixed strategy wt
% % i ∈ ∆(Si). At the end of the iteration each player i
% % observes the expected utility he would have received had he played any possible strategy x ∈ Si
% % .
% % More formally, let u
% % t
% % i,x = Es−i∼wt
% % −i
% % [ui(x, s−i)], where s−i
% % is the set of strategies of all but the i
% % th
% % player, and let u
% % t
% % i = (u
% % t
% % i,x)x∈Si
% % . At the end of each iteration each player i observes u
% % t
% % i
% % . Observe that
% % the expected utility of a player at iteration t is simply the inner product hwt
% % i
% % , u
% % t
% % i
% % i

% is there a bandit version?


% %http://www.economics.uci.edu/~ivan/asmb.874.pdf
% A modern Bayesian look at the multi-armed bandit

% %Efficient Algorithms for Adversarial Contextual Learning http://www.jmlr.org/proceedings/papers/v48/syrgkanis16.pdf




% Non-trivial two-armed partial-monitoring games are bandits
% %http://arxiv.org/abs/1108.4961

% % 4 proofs of gittins http://stat.haifa.ac.il/~gweiss/publications/fourbandits.pdf


% % The partial-monitoring games we consider are defined as follows: Two players interact with each
% % other in a sequential manner, Learner and Nature. In each time step Learner can choose one of N
% % actions, while Nature can choose one of M actions. We use the notation n = {1, . . . , n} for any
% % integer and denote the actions of both players by integers, starting from 1, so the action sets are N
% % and M. At the beginning of the game both Learner and Nature are given a pair of N ×M matrices,
% % G = (L, H), where L is the loss matrix and H is the feedback matrix. The elements ℓij of L are
% % real numbers and in fact we shall assume that they belong to the [0, 1] interval. The elements hij
% % of H could be chosen from any alphabet. However, for the sake of simplicity, and without loss of
% % generality (w.l.o.g.), we may assume that the elements of H are also real numbers. Now, still at
% % the beginning of the game, Nature decides about the sequence of actions (J1, J2, . . .) to be played.
% % These actions are kept private, i.e., they are not revealed to Learner. Nature’s actions will also be
% % called outcomes.
% % The game is played in discrete time steps. At time step t (t = 1, 2, . . .), first Learner chooses
% % an action It based on the information available to him up to time t. The choice of the action may
% % be randomized. Upon announcing his action, Learner gets the feedback hIt,Jt and suffers the loss
% % ℓIt,Jt
% % . The cycle is then repeated for time step t + 1. It is important to note that the loss suffered
% % is not revealed to Learner.




% % %http://people.csail.mit.edu/costis/bayesian/eva.pdf
% % Auctions as Games
% % • Simultaneous second price?
% % Christodoulou, Kovacs, Schapira ICALP’08
% % Bhawalkar, Roughgarden SODA’10
% % • Greedy Algorithm as an Auction Game
% % Lucier, Borodin, SODA’10
% % • AdAuctions (GSP)
% % Paes-Leme, T FOCS’10, Lucier, Paes-Leme + CKKK EC’11
% % • First price?
% % Hassidim, Kaplan, Mansour, Nisan EC’11
% % • Sequential auction?
% % Paes Leme, Syrgkanis, T SODA’12, EC’12



% % Some aspects of the sequential design of experiments - ‎Robbins - Cited by 1296
% % A problem in the sequential design of experiments - ‎Bellman - Cited by 238
% % Sequential design of experiments - ‎Chernoff - Cited by 257


% % TODO: frame as a partial monitoring games, two armed equivalence ref %http://arxiv.org/pdf/1108.4961v1.pdf 
% % Toward a classification of finite partial-monitoring games
% % Partial-monitoring games constitute a mathematical framework for sequential decision making problems with imperfect feedback: the learner repeatedly chooses an action, the opponent responds with an outcome, and then the learner suffers a loss and receives a feedback signal, both of which are fixed functions of the action and the outcome. The goal of the learner is to minimize his total cumulative loss. We make progress toward the classification of these games based on their minimax expected regret. Namely, we classify almost all games with two outcomes and a finite number of actions: we show that their minimax expected regret is either zero, , Θ(T2/3), or Θ(T), and we give a simple and efficiently computable classification of these four classes of games. Our hope is that the result can serve as a stepping stone toward classifying all finite partial-monitoring games.




% %example with personalized drug repositioning for the semi batch invitro approach http://journals.plos.org/plosone/article?id=10.1371/journal.pone.0135556



% %http://arxiv.org/pdf/1507.07191.pdf
% % In the on-line Explore & Exploit literature, central to Machine Learning, a central
% % planner is faced with a set of alternatives, each yielding some unknown reward. The
% % planner’s goal is to learn the optimal alternative as soon as possible, via experimentation.
% % A typical assumption in this model is that the planner has full control over
% % the experiment design and implementation. When experiments are implemented by a
% % society of self-motivated agents the planner can only recommend experimentation but
% % has no power to enforce it. Kremer et. al. [9] introduce the first study of explore
% % and exploit schemes that account for agents’ incentives. In their model it is implicitly
% % assumed that agents do not see nor communicate with each other. Their main result is
% % a characterization of an optimal explore and exploit scheme. In this work we extend [9]
% % by adding a layer of a social network according to which agents can observe each other.
% % It turns out that when observability is factored in the scheme proposed by Kremer et.
% % al. is no longer incentive compatible. In our main result we provide a tight bound on
% % how many other agents can each agent observe and still have an incentive-compatible
% % algorithm and asymptotically optimal outcome.

% %\subsection{Constructive Instituional Design} 
% %http://aiecon.tumblr.com/post/489827144/what-is-constructive-economics



% % Constructive Economics is the design, construction, implementation, and analysis of new economic structures. It’s about building things, making them real, and treating them as solid objects worthy of respect, admiration, study, and further improvement. Above all else: make it real.

% % Less Talk, More Rock

% % So what does making it real look like?

% % %Most profoundly, it involves the 

% % We do not reject equilibrium analysis, but do see it's role as limited to setting benchmarks which we might seek to achieve with strategies that have good properties out of equilibrium, in particular in the presence of less rational actors. A 

% %  and game theory. Game theory is only relevant inasmuch as it gives prescriptions for the strategy that you should be playing – like the minimax result for two-player zero-sum games and dominant strategies generally. To presume precisely how other people will behave is arrogant.

% % A canonical work in Constructive Economics would include the design, implementation, and analysis of a novel structure. Of course, such projects take a lot of time (and often, money) to implement, so making them real may not be directly feasible. But we should also embrace descriptions of practical systems that could be implemented, where here practical means a workable solution to an issue people are actually encountering. Research that might apply to auctions, doesn’t. At the same time, just because a market already exists shouldn’t exempt it from study. Markets should be regarded as natural phenomena, and their outputs should be subject to the Scientific Method. Just how do markets work, anyway?

% % Now let’s talk about what isn’t constructive:

% % Making real stuff is not the same as discarding un-real stuff. Work that describes what’s not possible should be treated warily. Oftentimes, such impossibilities are driven by modeling assumptions that don’t hold in practice. It’s much better to observe something actually happening in the real world and work back from that (e.g., The Rural Hospital Theorem)

% % Constructive Economics rejects the developing movement which argues that people are best modeled as perfectly rational agents that can’t solve NP hard problems. This is grafting two unrealistic paradigms on top of one another, producing a Frankenstein of shitty modeling.

% % Where Do We Go From Here?

% % Constructive Economics means embracing new fields. We need to pay more attention to details that are usually ignored, like interaction interfaces, which can have huge impacts on results. Furthermore, some of the most innovative economic structures of the last decade have emerged from outside the academic community, on Wall Street. Nobody in Computer Science writes papers on new kinds of mortgage bonds or interest rate swaps – but why not? Can’t we make a better VADM?

% % It also means abandoning some of the things we’ve traditionally regarded as sacrosanct. Rationality flies right into the garbage the moment you start seriously looking at the behavior of real people. Common knowledge of joint probability distributions? Half the people using your market don’t even know what a probability distribution is. Independent private valuations? Hell, even I’ve bid twice in an eBay auction. Worst-case bounds and hardness results? You might be surprised to find out how powerful CPLEX is.

% % There’s lots of overlap with these ideas and other parts of Econ/CS: Behavioral game theory and average-case analysis, to name a couple. What makes Constructive Economics different? Back to the first dictum: make it real. When I think about traditional, non-constructive, economic thinking, the image I find most appropriate is that of “lecturing birds on flying”. Birds already fly pretty well. Let’s start building airplanes.



% % %http://www.cs.cmu.edu/~sandholm/envy%20quotes%20for%20CAs.aaai10.pdf
% % The Vickrey-Clarke-Groves mechanism (VCG) is ubiquitous
% % in theoretical mechanism design. In the standard privatevalues
% % setting, it is the revenue-maximizing mechanism
% % among all incentive compatible individually rational effi-
% % cient mechanisms [Krishna and Perry, 1997]. Unfortunately,
% % in addition to being unwieldy to implement in practice, VCG
% % suffers from a number of pathologies [Rothkopf et al., 1990,
% % Sandholm, 2000, Ausubel and Milgrom, 2006, Rothkopf,
% % 2007]. These include revenue non-monotonicity, in which
% % adding another bidder can lower the seller’s revenue, and
% % receiving an arbitrarily small fraction of the revenue achievable
% % by posting prices.
% % To obtain higher revenues than VCG, one can explore
% % inefficient mechanisms, which leads to combinatorial
% % generalizations of the revenue-maximizing single-item
% % auction [Myerson, 1981]. Revenue-maximizing mechanisms
% % are unknown even for the (unrestricted) twoitem
% % setting, and in general a concise description of
% % the revenue-maximizing combinatorial auction cannot exist
% % (unless P=NP) because that design problem is NPcomplete
% % [Conitzer and Sandholm, 2004]. Some work has
% % been done on automated mechanism design for finding highrevenue
% % combinatorial auctions, but those approaches have
% % not been used for large numbers of items [Likhodedov and
% % Sandholm, 2004, 2005]. Even simple revenue-enhancemen
% % approaches like setting reserve prices require good knowledge
% % of a prior distribution over agent valuations, which may
% % or may not be available depending on the setting.
% % A different approach is to relax incentive compatibility.
% % One recent stream of research in non-incentive compatible
% % mechanisms has involved core-selecting combinatorial
% % auctions. Mechanisms of this class were used in the recent
% % successful spectrum license auction in the United Kingdom
% % [Cramton, 2008a,b, Day and Cramton, 2008]. These
% % mechanisms mitigate the poor revenue properties of the
% % VCG mechanism without necessarily subscribing to a firstprice
% % mechanism, which motivates significant underbidding.
% % Selecting an outcome in the core yields a host of desirable
% % properties that VCG lacks [Parkes, 2002], such as revenue
% % monotonicity and resistance to bidding using multiple
% % pseudonyms [Yokoo, 2006].
% % The word iterative has taken on a confusing double meaning
% % in combinatorial auction research. On the one hand,
% % the auction process itself can be iterative, in which bids
% % are solicited in a series of rounds until a termination condition
% % is reached (for instance, no agent submits a new bid).
% % This is the concept we explore in this paper. On the other
% % hand, given a set of bids, a solution may be produced iteratively,
% % for example, by raising the price of bundles in
% % ascending rounds in a specific way until reaching a point
% % in the core. Examples of this latter process include Parkes
% % [1999], Ausubel and Milgrom [2002], Wurman et al. [2004],
% % and Hoffman et al. [2006]. Unfortunately, these techniques
% % tend to be too slow to be used in an explicitly multi-round
% % auction, so these mechanisms work by inputting valuations
% % into proxy agents that bid on behalf of the auction participants.
% % Thus, price increases are a function of iterating on
% % the bids of these proxy agents rather than multiple iterative
% % rounds of buyers changing their valuations. As a consequence
% % these auctions are essentially one-shot. Thus they
% % do not achieve the main benefit of auctions that elicit the
% % bidders’ offers via multiple interactions: multi-interaction
% % mechanisms in effect give feedback to each bidder regarding
% % what information is (not) needed from her, thus reducing the
% % bidders’ deliberation effort in determining their own valuations.
% % For a review of preference elicitation in combinatorial
% % auctions see Sandholm and Boutilier [2006].
% % One attempt to get around the inability to solve for a
% % core solution quickly is the clock-proxy auction [Ausubel
% % et al., 2006]. It maintains (fast-to-compute) linear prices
% % over items through a number of bidding rounds before solv-




% % How Pervasive is the Myerson-Satterthwaite Impossibility?
% % Date of Original Version
% % 2009
% % Type
% % Conference Proceeding
% % Abstract or Description
% % The Myerson-Satterthwaite theorem is a foundational impossibility result in mechanism design which states that no mechanism can be Bayes-Nash
% % incentive compatible, individually rational, and not run a deficit. It holds universally for priors that are continuous, gapless, and overlapping. Using automated mechanism design, we investigate how often the impossibility occurs over discrete valuation domains. While the impossibility appears to hold generally for settings with large numbers of possible valuations (approaching the continuous case), domains with realistic valuation structure circumvent the impossibility with surprising frequency. Even if the impossibility applies, the amount of subsidy required to achieve individual rationality and incentive compatibility is relatively small, even over large unstructured domains.




% %http://arxiv.org/pdf/1505.00391.pdf
% % Learning and Efficiency in Games with Dynamic
% % Population
% % Thodoris Lykouris*
% % Department of Computer Science, Cornell University, Gates Hall, Ithaca, NY 14853, USA, teddlyk@cs.cornell.edu
% % Vasilis Syrgkanis
% % Microsoft Research, New York City, 641 Avenue of Americas, New York, NY 10011, vasy@microsoft.com
% % Eva Tardos ´ †
% % Department of Computer Science, Cornell University, Gates Hall, Ithaca, NY 14853, USA, eva@cs.cornell.edu
% % We study the quality of outcomes in repeated games when the population of players is dynamically
% % changing and participants use learning algorithms to adapt to the changing environment. Game theory
% % classically considers Nash equilibria of one-shot games, while in practice many games are played repeatedly,
% % and in such games players often use algorithmic tools to learn to play in the given environment. Learning in
% % repeated games has only been studied when the population playing the game is stable over time.
% % We analyze efficiency of repeated games in dynamically changing environments, motivated by application
% % domains such as packet routing and Internet ad-auctions. We prove that, in many classes of games, if players
% % choose their strategies in a way that guarantees low adaptive regret, then high social welfare is ensured,
% % even under very frequent changes. This result extends previous work, which showed high welfare for learning
% % outcomes in stable environments. A main technical tool for our analysis is the existence of a solution to
% % the welfare maximization problem that is both close to optimal and relatively stable over time. Such a
% % solution serves as a benchmark in the efficiency analysis of learning outcomes. We show that such stable
% % and near-optimal solutions exist for many problems, even in cases when the exact optimal solution can be
% % very unstable. We develop direct techniques to show the existence of a stable solution in some classes of
% % games. Further, we show that a sufficient condition for the existence of stable solutions is the existence of
% % a differentially private algorithm for the welfare maximization problem. We demonstrate our techniques by
% % focusing on three classes of games as examples: simultaneous item auctions, bandwidth allocation mechanisms
% % and congestion games


% % The goal of this paper is to understand the quality of outcomes of games and simple mechanisms
% % in a dynamic environment. The Internet allows for the repeated strategic interaction of many entities
% % with constantly changing parameters and participants. Primary examples of such interactions
% % include online advertising auction platforms, packet routing and allocation of cloud computing
% % resources. Understanding whether the constant change in these strategic environments can severely
% % damage the efficiency of the corresponding system, as compared to the hypothetical centralized
% % optimal, is of prime importance as these systems constitute the cornerstone of the online economy.
% % For example, advertising provides close to 90% of Google’s revenue (Google 2015).
% % Classical economic analysis of the interaction of strategic agents assumes that players reach a
% % stable outcome where all players are mutually best-responding to each others’ actions (or considers
% % mechanisms that are dominant strategy solvable). Dynamic environments, with high volume
% % interactions of small individual value or cost, such as packet routing or ad-auctions, are better
% % modeled as repeated games with learning players. Nash equilibria of the one-shot game correspond
% % to stable outcomes repeated in each iteration, where the players have no regret for their choice of
% % strategies. Hence, analyzing the quality of outcomes in repeated games via the price of anarchy
% % (Koutsoupias and Papadimitriou 1999) assumes that the repeated game reaches a stable, stationary
% % outcome. Such an analysis of price of anarchy of one-shot Nash equilibria has received large
% % attention in the past few years in both the computer science and operations research community
% % and in a plethora of application domains such as routing games (Roughgarden and Tardos 2002,
% % Correa et al. 2003), bandwidth allocation (Johari and Tsitsiklis 2004), strategic supply among
% % firms (Johari and Tsitsiklis 2011) and online ad-auctions (Caragiannis et al. 2015) (see e.g. Chapters
% % 17 to 21 of (Nisan et al. 2007) for a survey).
% % A more attractive model of player behavior in such repeated environments is to assume players
% % use a form of algorithmic learning. Modeling players as learners is especially appealing in online
% % auctions, as individual auctions provide very little value, costing only a few cents to a few dollars
% % each, so using experimentation to learn from the data is natural. Many advertisers use sophisticated
% % optimization tools or services to optimize their bidding, such as Bluekai1 or AdRoll2
% % .
% % It is well known that in most games natural game play does not lead to equilibria, under any
% % definition of “natural play” (see e.g. Chapter 7 of (Hart and Mas-Colell 2012)). In fact, results on
% % polynomial time computability of Nash equilibria of general games are mostly negative: finding
% % equilibria is computationally hard (see (Daskalakis 2009) for a survey).

% % Even with computational concerns aside, the game that the participants are playing at each
% % time-step and the participants they are playing against, can change at any time without even the
% % players realizing it or being able to form any distributional belief. Hence, even the concept of a
% % Nash equilibrium is debatable in such an adversarially evolving setting, as the players don’t even
% % have the information necessary to calculate their expected utility at each time-step. Instead they
% % observe their utility from the action they took or from any alternative action they could have taken,
% % only after the fact. In such an evolving setting, players can base their actions on past experience.
% % A particular class of learning behaviors, no-regret learning, emerged as a nice way to capture
% % the intuition that players learn to play appropriate strategies over time without necessitating
% % convergence to a stationary equilibrium. A stationary distribution that is also a no-regret learning
% % outcome corresponds to a Nash equilibrium of the one shot game, and in this sense, learning
% % outcomes generalize Nash equilibrium. More importantly, there are several simple and natural
% % algorithms that achieve the no-regret property (e.g. regret matching (Hart and Mas-Colell 2000),
% % multiplicative weight updates (Arora et al. 2012)). However, no-regret does not preclude the use
% % of possibly much more sophisticated tools, including using the above learning algorithms with
% % more complex benchmarks. Achieving small regret is a relatively simple expectation from bid
% % optimization tools.
% % Blum et al. (2006, 2008) consider regret-minimization as a model of player behavior in repeated
% % games, and study the average inefficiency of the outcome, coining the term price of total anarchy
% % for the worst-case ratio between the optimal objective value and the average objective value when
% % players use a no-regret algorithm. In a sequence of play all players achieve the no-regret property, if
% % and only if the empirical distribution of strategy vectors is a coarse correlated equilibrium, hence the
% % price of total anarchy is the ratio of the socially optimal welfare to the welfare at the worst coarse
% % correlated equilibrium. Roughgarden (2009) observed that many of the Nash equilibrium price
% % of anarchy bounds are shown via a proof technique he called smoothness, and such proofs easily
% % extend also to show bounds on the quality of coarse correlated equilibria. Syrgkanis and Tardos
% % (2013) extend smoothness to simple mechanisms, such as independent item auctions.
% % However, this learning outcome analysis is based on the strong assumption that the underlying
% % environment and player population is stable. The reason for this requirement is easy to understand:
% % with the game and the players stable, there is a fixed optimal solution, and a fixed strategy, that
% % each player i would need to play (action a
% % ∗
% % i
% % ) as his or her part for achieving the optimum. To
% % guarantee high social welfare via the smoothness approach, all we need is that each player i doesn’t
% % regret not playing this optimal action a
% % ∗
% % i
% % . No-regret learning guarantees exactly this; player i will
% % not regret any fixed strategy with hindsight, including strategy a
% % ∗
% % i
% % . However, online environments
% % are typically not stable.


% % In this paper, we study learning outcomes in games with a dynamically changing player population.
% % As stated in (Fudenberg and Levine 1998, p. 4), the fact that players extrapolate across
% % games they view as similar is an important reason learning has relevance in a real-world situation.
% % A repeated game with an evolving population is exactly a setup where players are asked to play
% % repeatedly in similar games. Rather than aiming to predict the exact outcome, our goal is to predict
% % properties of outcomes, such as their efficiency, i.e. the price of anarchy.
% % In a changing game environment, we need a slightly stronger notion of regret minimization.
% % No-regret learning aims to select strategies that do at least as well on the average over a sequence
% % of steps as the best single strategy would have done in hindsight. With the game environment and
% % population changing, a single best strategy in hindsight gives a really weak benchmark. Players,
% % using good learning algorithms, should be able to adapt to the changing environment, and such
% % adaptation may be very useful with the population changing over time. For example, in the context
% % of routing games, a player with many route options, may want to adjust their route choices
% % depending which part of the network is more congested, or in auction games, a player may want
% % to bid for items that are less in demand.
% % Hazan and Seshadhri (2007) formally introduced the stronger notion of adaptive regret that we
% % will use, bounding the average regret over any sub-interval of steps [τ1, τ2), compared to a single
% % best action over this interval in hindsight. The study of adaptive learning goes back much further:
% % the work of Lehrer (2003) and Blum and Mansour (2007) studied generalizations of adaptive regret
% % prior to (Hazan and Seshadhri 2007). Clearly short intervals will result in relatively high regret with
% % any learning algorithm, but adaptive learning algorithms guarantee for the player that the cumulative
% % regret grows sub-linearly with the length of the interval. Most adaptive learning algorithms
% % are constructed by modifying classical no-regret learning algorithms to stop relying too heavily
% % on experience from the distant past. We believe that such adaptive learning is a better model of
% % behavior when strategic agents (such as bidders in online auctions) use sophisticated optimization
% % tools. The current best adaptive learning algorithm is a natural adaptation of the classical Hedge
% % algorithm, AdaNormalHedge, due to Luo and Schapire (2015). With this framework in mind, we
% % ask the following main question:
% % How much rate of change can a system admit to sustain approximate efficiency, when its
% % participants are adaptive learners?
% % Our Results. We show that in large classes of games, if players choose their strategies in a
% % way that guarantees low adaptive regret, this ensures high social welfare, even under surprisingly
% % high turnover. To model a changing environment we consider a dynamic player population where
% % between every pair of iterations each player leaves independently with a (small) probability p and
% % is replaced by an arbitrary new player, implying that in expectation a p fraction of the population



% % % http://people.csail.mit.edu/smweinberg/thesis.pdf
% % Abstract
% % In traditional algorithm design, no incentives come into play: the input is given, and your algorithm
% % must produce a correct output. How much harder is it to solve the same problem when the input
% % is not given directly, but instead reported by strategic agents with interests of their own? The
% % unique challenge stems from the fact that the agents may choose to lie about the input in order to
% % manipulate the behavior of the algorithm for their own interests, and tools from Game Theory are
% % therefore required in order to predict how these agents will behave.
% % We develop a new algorithmic framework with which to study such problems. Specifically, we
% % provide a computationally efficient black-box reduction from solving any optimization problem on
% % “strategic input,” often called algorithmic mechanism design to solving a perturbed version of that
% % same optimization problem when the input is directly given, traditionally called algorithm design.
% % We further demonstrate the power of our framework by making significant progress on several
% % long-standing open problems. First, we extend Myerson’s celebrated characterization of single
% % item auctions [Mye81] to multiple items, providing also a computationally efficient implementation
% % of optimal auctions. Next, we design a computationally efficient 2-approximate mechanism
% % for job scheduling on unrelated machines, the original problem studied in Nisan and Ronen’s paper
% % introducing the field of Algorithmic Mechanism Design [NR99]. This matches the guarantee of
% % the best known computationally efficient algorithm when the input is directly given. Finally, we
% % provide the first hardness of approximation result for optimal mechanism design.


%can we make use of http://arxiv.org/pdf/1606.06244v2.pdf
% log barrier regularization with importance sampling that guarantees fast convergence of O(d log T/ǫ) 
% note they make no 




%https://arxiv.org/pdf/1201.6429v2.pdf
% Bounding the inefficiency of outcomes in generalized second price auctions.
% . We show that the welfare generated by the auction in any equilibrium of bidding behavior is
% at least a 1
% η
% -fraction of the maximum achievable welfare (i.e., the welfare the auction could generate knowing
% the player types and quality factors in advance). The value of η measures the robustness of an auction
\cite{caragiannis2015bounding}
% The Generalized Second Price (GSP) auction is the primary auction used for monetizing the use of the Internet. It is well-known that truthtelling is not a dominant strategy in this auction and that inefficient equilibria can arise. Edelman et al. (2007) [11] and Varian (2007) [36] show that an efficient equilibrium always exists in the full information setting. Their results, however, do not extend to the case with uncertainty, where efficient equilibria might not exist.

% In this paper we study the space of equilibria in GSP, and quantify the efficiency loss that can arise in equilibria under a wide range of sources of uncertainty, as well as in the full information setting. The traditional Bayesian game models uncertainty in the valuations (types) of the participants. The Generalized Second Price (GSP) auction gives rise to a further form of uncertainty: the selection of quality factors resulting in uncertainty about the behavior of the underlying ad allocation algorithm. The bounds we obtain apply to both forms of uncertainty, and are robust in the sense that they apply under various perturbations of the solution concept, extending to models with information asymmetries and bounded rationality in the form of learning strategies.

% We present a constant bound (2.927) on the factor of the efficiency loss (price of anarchy) of the corresponding game for the Bayesian model of partial information about other participants and about ad quality factors. For the full information setting, we prove a surprisingly low upper bound of 1.282 on the price of anarchy over pure Nash equilibria, nearly matching a lower bound of 1.259 for the case of three advertisers. Further, we do not require that the system reaches equilibrium, and give similarly low bounds also on the quality degradation for any no-regret learning outcome. Our conclusion is that the number of advertisers in the auction has almost no impact on the price of anarchy, and that the efficiency of GSP is very robust with respect to the belief and rationality assumptions imposed on the participants.

% The now-standard mechanism for resolving online search advertisement requires that each advertiser
% places a bid that represents the maximum she would be willing to pay if a user clicked her ad. These bids
% are then resolved in an automated auction whenever ads are to be displayed. By far the most popular bidresolution
% method currently in use is the Generalized Second Price (GSP) auction, a generalization of the
% well-known Vickrey auction. In the GSP auction, there are multiple ad “slots” of varying appeal (e.g. slots
% at the top of the page are more effective). In two seminal papers Edelman et al. [11] and Varian [36] propose
% a simple model of the GSP auction that we will also adopt in this paper. They observe that truthtelling is
% not a dominant strategy under GSP, and GSP auctions do not generally guarantee the most efficient outcome
% (i.e., the outcome that maximizes social welfare). Nevertheless, the use of GSP auctions has been extremely
% successful in practice. This begs the question: are there theoretical properties of the Generalized Second
% Price auction that would explain its prevalence? Edelman et al. [11] and Varian [36] provide a partial
% answer to this question by showing that, in the full information setting, a GSP auction always has a Nash
% equilibrium that has same allocation and payments as the VCG mechanism. [11] and [36] give only informal
% arguments to justify the selection of envy-free equilibria.
% We argue that the Generalized Second Price auction is best modeled as a Bayesian game of partial
% information. Modeling GSP as a full information game assumes that each auction is played repeatedly with
% the same group of advertisers, and during such repeated play the bids stabilize. The resulting stable set of
% bids is well modeled by a full information Nash equilibrium. The analyses of Edelman et al. [11] and Varian
% [36] provide important insight into the structure of the GSP auction under this assumption. However, the set
% and types of players can vary significantly between rounds of a GSP auction. Each query is unique, in the
% sense that it is defined not only by the set of keywords invoked but also by the time the query was performed,
% the location and history of the user, and many other factors. Search engines use complex machine learning
% algorithms to select the ads, and more importantly to determine appropriate quality scores (or factors) for
% each advertiser for a particular query, and then decide which advertiser to display. This results in uncertainty
% both about the competing advertisers, and about quality factors. We model this uncertainty by viewing the
% GSP auction as a Bayesian game, and ask: what are the theoretical properties

% Our
% framework of semi-smooth games is an extension of Roughgarden’s [31] smoothness framework, that allows
% dealing with correlated distributions. Correlated distributions are an important feature of the GSP model,
% especially when modeling quality factors, as the same facts affect clickability and hence the quality factors
% for all advertisers. (For instance, an ad shown to a bot will not get a click independent of the advertiser.)


% For mechanisms that are not dominant strategy truthful, like GSP auctions, price of anarchy analysis is a
% powerful tool for quantifying the potential loss of efficiency at equilibrium. We conduct this analysis both in
% a full information setting without uncertainty (in which the price of anarchy is surprisingly small, indicating
% a loss of at most 22% of the welfare), but also in a setting with uncertainty and a very general information
% structure, in which we prove that the price of anarchy is still bounded by a small constant. This shows
% that while the GSP auction is not guaranteed to be efficient, it is a reasonably good design, as remarkably,
% the welfare loss of these auctions is bounded by a value that does not depend on the number of players,
% the number of advertisements for sale, or the prior distributions on player types. In contrast, the variant
% of the Generalized Second Price auction that orders advertisers by their bid ignoring quality factors, which
% has been historically used by Yahoo!, results in a quality loss proportional to the range of quality factors,
% while randomly assigning advertisers to slots can result in a loss of efficiency proportional to the number of
% advertisers.
% One feature of our results is that they hold for a variety of models regarding the rationality and the beliefs
% of the players. This robustness is particularly important in large-scale auctions conducted over the Internet,
% where assumptions of full information and/or perfect rationality of the participants are unreasonably strong.



%for furhter bandit with self interested advisors

% Efficient Partial Monitoring with Prior Information
% Hastagiri P Vanchinathan
% Dept. of Computer Science
% ETH Zurich, Switzerland ¨
% hastagiri@inf.ethz.ch
% Gabor Bart ´ ok´
% Dept. of Computer Science
% ETH Zurich, Switzerland ¨
% bartok@inf.ethz.ch
% Andreas Krause
% Dept. of Computer Science
% ETH Zurich, Switzerland ¨
% krausea@ethz.ch
% Abstract
% Partial monitoring is a general model for online learning with limited feedback: a
% learner chooses actions in a sequential manner while an opponent chooses outcomes.
% In every round, the learner suffers some loss and receives some feedback based on
% the action and the outcome. The goal of the learner is to minimize her cumulative
% loss. Applications range from dynamic pricing to label-efficient prediction to dueling
% bandits. In this paper, we assume that we are given some prior information about the
% distribution based on which the opponent generates the outcomes. We propose BPM, a
% family of new efficient algorithms whose core is to track the outcome distribution with
% an ellipsoid centered around the estimated distribution. We show that our algorithm
% provably enjoys near-optimal regret rate for locally observable partial-monitoring
% problems against stochastic opponents. As demonstrated with experiments on synthetic
% as well as real-world data, the algorithm outperforms previous approaches, even for very
% uninformed priors, with an order of magnitude smaller regret and lower running time.